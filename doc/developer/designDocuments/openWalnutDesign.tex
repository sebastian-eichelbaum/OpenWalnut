\documentclass[a4paper,12pt]{scrbook}
\usepackage[bookmarks=true,colorlinks=true]{hyperref} 
\hypersetup{
 pdfauthor = {Alexander Wiebel},
 pdftitle = {OpenWalnut Design},
 pdfsubject = {Design Document},
 pdfkeywords = {Visualization System, Medical Visualization},
 pdfcreator = {LaTeX with hyperref package},
 pdfproducer = {pdflatex}}

\usepackage[latin1]{inputenc}
\usepackage{color}
\usepackage{graphicx}
\usepackage{pst-grad,pstricks}
\usepackage{amsmath}
\usepackage{amssymb}
\usepackage{amsfonts}
\usepackage{subfigure}
\usepackage{wrapfig}
\bibliographystyle{alpha} 

\newcommand{\todo}[1]{\textcolor{white}{\colorbox{red}{\\ Noch zu %
      tun:}}\ \ #1 \textcolor{red}{\colorbox{red}{I}}\\\ \\\ }
\newcommand{\here}{\textcolor{white}{\colorbox{red}{\\\LARGE HIER}} }

\begin{document}
\titlehead{
  OpenWalnut Project\\
  www.openwalnut.org}

\subject{Documentation}
\title{
%  \includegraphics{Pictures/FAnToM-logo}\\[.2cm]
  OpenWalnut Design
}
\author{ Alexander Wiebel }
\publishers{Leipzig}
\maketitle

\tableofcontents

\chapter{GUI}
The GUI module is the only part where we allow gui-toolkit related code. For example, this means that the use of any QT class is prohibited outside the GUI module.
\chapter{GE (Graphics Engine)}
\todo{This is still wrong}
The GraphicsEngine module is the only part where we allow OpenGL or shader code. 
 There are different kind of objects being treated in the GraphicsEngine: 
Graphics primitives like triangles, lines, ... 
Textures, at least 2D and 3D 
3D textures often are the representation of data from a data set in the DataHandler that is needed for rendering.
\chapter{DataHandler}
The sections in this chapter are organized according to different
types od recording or imaging data.
\section{EEG}
\subsection{Note}
 sread in BIOSIG C++ toolbox provides start and length parameters ... these should be used as EEG might not fit or at least is not intended to be completely in the main memory. This should also be reflected in the design of the data structure.
 Is segment-wise loading a good solution?
\section{Taken from Wiki}
\todo{Has to be fixed as it was 
This page describes the main ideas of the architecture of our /src/dataHandler. 
WDataHandler is the main entry point to the module. It contains all
the data sets. Each of the data sets is encapsulated by a WDataSet
object. 

 WDataSet is the base class for a number of subclasses representing
 different types of data sets, e.g. 
\begin{itemize}
\item WDataSetSingle - A simple single data set. 
\item WDatSetTimeDependent - A data set that encapsulates a number of
  data sets for different time steps. 
\end{itemize}

WDataSet has a WMetaInfo 

A WDataSetSingle is a WDataSet and has a WValueSet 
and a WGrid 

A WDataSetTimeDependent is a WDataSet has more than one WValueSet and
at least one WGrid (at the moment, i.e. 2009-05-13 we expect only one
WGrid) 

 A WMetaInfo has a number of attribute describing a WDataSet, e.g.: 
Name of data set 
FileName: Name of the file the data was loaded from. 
... 
 A WMetaInfoTimeDependent is a WMetaInfo and has a 
a number of additional attributes. 

 A WValueSet has subclasses for different types of values. 

 A data set has a low level access for the GraphicsEngine? to ensure
 performance. This access is realized by a function similar to 
\begin{verbatim}
T* getValuesPointer< class T >()
\end{verbatim}
 and code to get the access then look similar to 
\begin{verbatim}
float *myValuePointer;
myValuePointer = getValuesPointer<float *>()
\end{verbatim}
Data can not be manipulated!  
 

\end{document}
